%% !TEX encoding = UTF-8 Unicode 
%% Author: Maarten Steenhagen / organisation: UCL
%% Created on 2012-04-25.
%% Copyright (c) 2012. All rights reserved.
\documentclass[a4paper,11pt, twoside]{article}
\usepackage{pja-article}

\thejournal{Postgraduate Journal of Aesthetics}{PJA} 
\thevolume{9}
\theissue{1}
\themonth{January}
\theyear{2012}
\thejournalinfo{http://www.pjaesthetics.org/}

\thehtitle{Literature and Sincerity} % Title in the header (no caps is fine)
\theauthor{Katy Thomas} % THESE ARE USED FOR LABELLING AS WELL
\theaffil{University of Warwick}
\thecontactd{~} % Currently not used
\theemail{k.thomas@warwick.ac.uk} % THESE ARE USED FOR LABELLING AS WELL

\firstpage{13} % Sets the first page number

\proof % Disable on final version

\begin{document}
\pdfdetails % Required to output author/title information in .PDF
	
\maketitle{literature and sincerity}% No caps should be used

\section{Introduction}
This paper addresses the relationship between literature and knowledge. The anti-cognitivists Peter Lamarque and Stein Haugom Olsen have suggested that the truth- value of the propositional content of a text is contingent upon its specifically \emph{literary} value \autocite{lamarque1994aa}. I think that they are right, and will spend the first section outlining the details of their theory. Nonetheless, I will then proceed to argue \emph{for} a cognitivist theory of literature. My thesis is that we value the literary work not for its `truth' ordinarily understood (in the scientific, maximally objective sense which Lamarque and Olsen invoke), but for its sincerity---that is, for its distinctively accurate and fine-grained expression of the states of mind of its author.

I postulate that this sincerity is the endeavour to convey accurately one’s states of mind, beliefs and emotions. The sincere subject, therefore, is concerned with enabling knowledge of his or her subjective experience. In other words, sincerity is related to the ability to express oneself successfully. The process of thematising a subject in a literary work is an intricate form of expression. The truth-value of literature, I suggest, is bound up with its nature as an expressive art.

\section{Literary appreciation}
To illustrate Lamarque and Olsen’s notion of literary appreciation, it is worth establishing what this activity is \emph{not}. These theorists do not deny that learning, in the most conventional sense, from a literary text is perfectly common:
\begin{quote}
\emph{Of course} readers can pick up information about people, places and events from works of fiction; \emph{of course} readers can learn practical skills, historical facts, points of etiquette, insights into Regency England, etc., from literary works; \emph{of course} writers of literary fiction often offer generalizations about human nature, historical events, political ideologies, and so forth, in their works; \emph{of course} what readers take to be true (in the world) will affect how they respond to literary works, including how they understand the works; \emph{of course} readers often need to have background knowledge of a cultural, psychological, or historical kind, even moral or philosophical preconceptions, to understand some literary works.\autocite[p. 4-5]{lamarque1994aa}\end{quote}
However, to truth-evaluate a text in the ways mentioned is to treat it as something other than literature. I agree that it is helpful and interesting to distinguish the literary from other modes of discourse, whether scientific, mathematical or philosophical. In order to take up what Lamarque and Olsen term the ``literary stance'' towards a text, one must engage with the text \emph{thematically}.

A work's theme ``emerges from the subject it has, the way in which the subject is presented, the rhetorical features used in its presentation, and the structure which it is given''\autocite[p. 408]{lamarque1994aa}. Rather than truth-assessing a text's propositional content, the reader makes a web of connections between elements of content and formal features of the work to elicit what the text is really `about'. This is a dynamic, speculative, yet potentially highly informed endeavour; a thematic reading must justify itself on the basis of thorough knowledge of the work.

The reader, then, apprehends the literary work by means of applying a ``thematic concep'' (examples given by Lamarque and Olsen in their own reading of Euripedes' Hippolytus include ``freedom, determinism, responsibility, weakness of will, continence/incontinence, sympathy, guilt'') to its diversity of elements (the framing of the narrative by the monologues of goddesses, the characterisation of the three protagonists, the imagery of uncleanliness, the role of the chorus, and so on)\autocite[p. 402]{lamarque1994aa}. Specific elements only take on significance and meaning in the context of the work as a coherent whole.

If we feel that in this process of thematic interpretation we have really \emph{learnt} something about the concept at hand, this learning, again, is incidental to our appreciation. It is not a part of the work’s literary value that it actually improves our concept of, for example, freedom, though it fills that concept with a complex and idiosyncratic content for the duration of our reading experience.``Literary appreciation'' claim Lamarque and Olsen, ``is concerned with the application of a set of thematic concepts to a particular literary work,'' but is utterly unconcerned with ``any further reality to which these concepts might be applied in their other uses.''\autocite[p. 409]{lamarque1994aa}

Lamarque and Olsen provide what seems to me an accurate description of literary engagement (as the activity of thematic unification of form and content) that helps us to distinguish literature from other modes of discourse. I agree that the straightforward acquisition of propositional truth (as they characterise it) \emph{is} irrelevant to the literary value of a text, as is the acquisition of some `objectively' more accurate concept. Given this, what is my objection to their anti-cognitivist argument?

I want to suggest that thematic appreciation, exactly as Lamarque and Olsen characterise it, can in \emph{itself} constitute a learning experience; where we learn not about the thematic concept itself, but about a cognitive-emotional experience the author has had involving such a concept. The author is the agent who makes available the often surprising connections we re-make in thematic appreciation: in this process of discovery we encounter the mind of another and the distinctive patterns this mind imposes upon the world. In order to explore in more depth how and what we learn from thematic appreciation, I must first interrogate Lamarque and Olsen’s summary dismissal of the relationship between truth and sincerity.

Early in their argument, Lamarque and Olsen take care to sharply distinguish sincerity from truth-telling:
\begin{quote}
[I]f `true' as applied to works of literature is used to mean `sincere', as it sometimes is, then the claim that literary works can be true or false amounts to no more than the claim that they can be sincere or insincere. There is always a danger in this debate of talking at cross purposes \ldots `is sincere' \ldots [is] not substitutable for `is true' \ldots \autocite[p. 5-8]{lamarque1994aa}\end{quote}Though they acknowledge that ``a sincere judgement aims for truth'', they argue that ``we cannot accept truth as equivalent to sincerity'' because ``sincere judgements can sometimes be false.''\autocite[p. 10]{lamarque1994aa} I might say, in all sincerity, `Eduardo Galeano is a Colombian writer', although Galeano is actually Uruguayan. I am being sincere, yet I fail to tell the truth. I suggest, however, that this is too simplistic an account of the relationship between sincerity and truth-telling. Sincerity is necessarily truthfulness \emph{about} one's own beliefs, feelings and so on; so what we gain from a successfully sincere utterance is knowledge of the speaker's state of mind, for example, the belief that Galeano is Colombian. This knowledge is valuable in its own right. Any extended exploration of sincerity itself is absent from Lamarque and Olsen's work. I will make a preliminary attempt at such an endeavour in this paper.

\section{Sincerity and self-knowledge}
In a recent paper, Richard Moran describes a dominant conceptual model for `sincerity' which I shall refer to as the `transparency' model. On this model, true communion with the mind of the other can only take place in a disembodied universe, where necessarily imperfect verbal and physical means of expression are no longer necessary. He refers to a passage from Plotinus to flesh out the commitments of this model:
\begin{quote}
We certainly cannot think of them [souls free of the body], it seems to me, as employing words when, though they may occupy bodies in the heavenly region, they are essentially in the Intellectual \ldots There can be no question of commanding or of taking counsel; they will know, each, what is to be communicated from another, by present consciousness. Even in our own case here, eyes often know what is not spoken; and There all is pure, every being is, as it were, an eye, nothing is concealed or sophisticated, there is no need of speech, everything is seen and known.\footnote{Plotinus, (Stephen McKenna, trans.) quoted in Moran \cite*[p. 325]{moran2005aa}.}\end{quote}In this utopia, we have direct, unmediated access to the mind of the other, without any barriers. If this kind of telepathic communion is our communicative ideal, as Moran points out, the value of \emph{sincerity} is located in the assumption that it takes us as close as we can come to this access: ``sincerity matters to speech because its presence is our guarantee that what the speaker says is an accurate representation of what he actually believes.''\autocite[p. 356]{moran2005aa}

Moran himself calls the transparency model into question. First, he suggests that it is important to distinguish between two modes of expression:
\begin{enumerate}
\item Expression in the impersonal sense as the unintentional manifestation by subject A of some attitude or state of mind.
\item Expression in the personal sense as the intentional act of subject A directed to another.\autocite[p. 335]{moran2005aa}
\end{enumerate}Personal expression is reliant upon A's awareness of the ``attitude or state of mind'' present in 1; it involves self-knowledge. Moreover, it involves an intention to make this attitude or state of mind available for others to learn about. As Moran puts it, A can only personally express ``his fear or his belief if he is aware of it and means to make it manifest.''\autocite[p. 338]{moran2005aa}

Even an ideally sincere utterance, Moran argues, does not give us unmediated access to the other's beliefs. What it informs us about is the speaker's self-knowledge of those beliefs, and the fact that the speaker wants us to know about them. The sincere speaker, then, ``does not simply lay his beliefs open to view (which can happen even with respect to an unconscious belief of his), something for the hearer to make of what he will, but rather registers his consciousness \ldots of the belief itself''. In this way, a sincere utterance relies upon the speaker's self-knowledge, which may be faulty. It gives us access not to the way the speaker is, but the way she takes herself to be.

Moran goes on to claim, therefore, that ``the demands of sincerity are [thus] weaker than the demands of accurate presentation of one’s beliefs or other attitudes''\autocite[p. 340]{moran2005aa}. An ideally sincere utterance offers us epistemic access only to the speaker's perception of her states of mind, not to the states of mind themselves. Though I sincerely say to my friend, `I’m angry with my sister because she ate the last Jaffa Cake', my anger with my sister might actually, without me being aware of it, derive from the preferential treatment she receives from my parents. My sincere utterance, while fulfilling `the demands of sincerity', offers no access to this state of mind due to my limited self-knowledge.

I would like to try a different tack and suggest that if we are talking in terms of demands upon us, in terms of a target one is trying to meet, surely when we try to be sincere we simply \emph{are} trying to accurately present our own states of mind: an activity which comprises the further commitments towards self-discovery such an aim entails. I think the problem with Moran's argument is that he assumes that sincerity is characterised by a kind of ``spontaneous'' immediacy. In his account, one does not have to think about how to be sincere, one simply is so ``naturally''\autocite[p. 340]{moran2005aa}. I want to suggest that such a conception is inadequate.

In practice, it seems commonplace to worry about one's ability to be properly sincere---to get across one’s state of mind---particularly when dealing with complex or troubling feelings and thoughts which are difficult to express. Sincerity, on my account, is something \emph{difficult}: an inseparably ethical, cognitive and aesthetic endeavour. It involves an intentional honesty with oneself, and is greatly aided by a level of facility with means of expression.

For my purposes here, if a constitutive aim of sincerity is to impart the truth about one's actual beliefs, attitudes and experience, as I believe it is, then the sincerity endeavour thus comprises a reasonable effort towards self-knowledge. The ``demands of sincerity''---that is, what is necessary for someone to count as sincere---include the demand to understand oneself. Genuine (as opposed to merely putative) sincerity requires, if not perfect self-knowledge, then at least the epistemically active effort to know oneself. We will explore below how this effort towards self-knowledge is related to the activity of expression.

\section{Expression and self-knowledge}
Moran seems to conceptualise personal expression as the deliberate externalizing of `something' (an inner state) which is already complete, noting that this also reveals the subject's self-knowledge of this something and her conscious intention to make it manifest to another. For R. G. Collingwood, however, the self-knowledge to which Moran refers is actually arrived at through the tricky process of externalisation. The phenomenon inside is transformed in this process, from something inchoate and vaguely felt to something precisely cognised. Collingwood's `expression' most closely resembles a kind of \emph{clarification}, and thus has a strongly cognitive orientation.

Artistic expression is centrally `'`the \emph{work} of getting clear about one’s own thoughts and feelings, of transforming muddle into clarity''\autocite[p. 6]{ridley1998aa}. This work takes place through engagement with a medium, whether words, gestures, paints, sounds or stone. In writing a line of poetry, or painting a picture, the artist is involved in a distinctive kind of purposeful activity. As Aaron Ridley has pointed out in his commentary on Collingwood, the kind of activity at hand calls into question standard accounts of purposive action. The aim of the activity is not specifiable independent of the activity itself: ``[the poet] can give no prior specification of what the right line will be; but when he’s got it, he knows.''\autocite[p. 29]{ridley1998aa} The process of engaging with the medium is itself a process of experimentation and discovery: both of the nature of one’s own feeling and ‘the possibilities of the medium’ in which this feeling is expressed.\footnote{Ridley \cite*[p. 28]{ridley1998aa}. Ridley points out that such a description also applies to other activities that we usually think of as more everyday than art-making, such as trying to say clearly what one means.}

Collingwood distinguishes between the description of an emotion and the expression of it. ``To say `I am angry','' claims Collingwood, ``is to describe one’s emotion, not to express it.''\autocite[p. 113]{collingwood1938aa} He suggests that when we describe a feeling as, for example, anger, we bring that feeling into a general category: we `bring it under a concept' and `classify it'. We `call it a thing of such and such a kind'\autocite[p. 113]{collingwood1938aa}. This descriptive activity has its own particular cognitive usefulness, in that `it groups a particular thing together with other things of the same general sort'\autocite[p. 29]{ridley1998aa}.

However, in the process of expressing a feeling of anger, we do not necessarily use the word `anger' at all. Our concern when undertaking this activity is to cognise fully the particular, idiosyncratic content of this particular emotion, which might include all kinds of disparate phenomena and intentional objects. In ``grappling with the problem of expressing a certain emotion,'' Collingwood argues, we say to ourselves ```I want to get this clear'. It is no use \ldots to get something else clear, however like it this other thing may be. Nothing will serve as a substitute.'' When undertaking the activity of expression, the subject ``does not want a thing of a certain kind, he wants a certain thing.''\autocite[p. 114]{collingwood1938aa} Expression ``marks the distinction \ldots between different things that might be described in the same way. To express something \ldots is to clarify it in all its particularity.''\autocite[p. 29]{collingwood1938aa}

It is also important to note that in Collingwood's theory, expression is distinct from what we think of as communication as such. One expresses a feeling, not necessarily in order to impart knowledge of one’s experience to another, but to gain self-knowledge: an act of expression is addressed, `primarily to the speaker himself', and only `secondarily to anyone who can understand'.\autocite[p. 110-11]{collingwood1938aa} In an encounter with an expressive artefact, the spectator ideally gains the same knowledge (of the artist's subjective experience) that the artist did in making it. It is the knowledge acquired by the spectator which is the chief subject of this paper, though discussion of the potential cognitive value of certain literary works must surely benefit from taking into account the experiences of both author and reader.\footnote{For Collingwood, the cognitive value of art for its audience lay primarily not in its capacity to reveal another's experience, but in its capacity to foster self-knowledge for the spectator. The artwork functions as an exemplary expression: ``[W]hen someone reads and understands a poem, he is not merely understanding the poet’s expression of his, the poet’s, emotions, he is expressing emotions of his own in the poet’s words, which have thus become his words'' \autocite[p. 283]{collingwood1938aa}. In fact, I agree that part of the reason we value art involves its capacity to teach us how to better express ourselves; however, for the purposes of this paper I want to focus exclusively on the access that the literary work might give us to another's experience, without taking the further step of examining the impact that knowledge might have upon our own self-knowledge.}

I want to suggest that if expressive facility with a medium is important to the acquisition of self-knowledge, then it is also important to the sincerity endeavour, given the relationship between self-knowledge and sincerity postulated in the previous section. Putting aside for now any debates around Collingwood's theory of emotions, I want to try using his model of expression to shed light on literary thematics and the cognitive value of thematic reading.

\section{Literary thematics}
When, say, `freedom' is thematised in a text, this does not involve explicit referential, propositional statements about freedom that can be easily identified in the text and truth-evaluated. It does not consist of a \emph{description} of freedom. As we have seen, it involves a whole host of differentiated yet linked formal and semantic constituents that serve to flesh out the content of the author's emotional-cognitive experience of the concept of freedom. It consists, therefore, in the author’s artistic \emph{expression} of his or her feelings about freedom.

This means that the artist is first and foremost concerned with the truth, not about freedom, but about her own feelings. As Collingwood puts it, a ``poet who is disgusted with life today, and says so, is not saying that he undertakes to be still disgusted tomorrow. But it is not any the less true that, today, life disgusts him. His disgust may be an emotion, but it is a fact that he feels it; the disgustingness of life may be an appearance, but the fact of its appearing is a reality.'' Art attempts to discover ``individual fact[s]'' about subjective experience.\autocite[p. 288]{collingwood1938aa}

Let me present an example: Eduardo Galeano's \emph{Memory of Fire} is a trilogy that takes `community' as one of its central themes. The three volumes comprise a literary history of Latin America from pre-Columbian times to the end of the twentieth century. It tells this history from a left-wing, revolutionary perspective, through hundreds of vignettes set all over the continent and narrated by a variety of characters. According to our reflections so far, in reading \emph{Memory of Fire} we `overhear', through thematic reading, Galeano’s expression of his feelings on the topic.

In accordance with Lamarque and Olsen's theory, when we elicit the theme of `community' from Galeano's work, we do not only extract explicit propositional statements from his text that make reference to this topic. We justify our identification of this theme, as a topic the text is `about', by pointing to a host of different constituents of the work. Such constituents, of course, might include explicit statements, which inform our reading and act as useful signposts to other elements of the text: ``We don’t like your custom of every man for himself rather than helping each other'', for example, uttered by the Guaraní Indians to a group of colonisers in the first volume, might constitute one of our first clues to the presence of this theme.\autocite[p. 35]{galeano1985aa}

As we read, we might begin to find significance in Galeano's structuring of his narrative in short, discrete vignettes involving a plethora of characters, with no single character followed consistently or continuously. Recurrent images or motifs which seem to be linked metaphorically to the thematic concept might begin to spring to our attention. The epigraph to the first volume of the trilogy is an African proverb which reads ``The dry grass will set fire to the damp grass.'' This phrase is echoed in a verbatim statement by a revolutionary in the second volume: ``Let’s set fire to the dry grass.''\autocite[p. 113]{galeano1987aa} What might this repeated image cause us to reflect upon? An area of dry, brown grass is undernourished, while the damp grass adjacent to it is well-fed and flourishing. This, however, makes the area of dry grass dangerous, as it is increasingly liable to catch fire; a fire which will spread to the healthy green grass. In the mouth of the revolutionary, the phrase is used to describe the perilous relationship of poor and rich in an unequal community.

We may begin to hold this image in our mind as we read, and link it to the prevalent natural imagery throughout the vignettes. ``Great fortunes in a few hands'' are ``stagnant waters that do not bathe the earth.''\autocite[p. 101]{galeano1987aa} ``A mestizo language grows'' in enslaved Nicaragua, as ``fiery chili from the imagination of a people making fun of its masters.''\autocite[p. 250]{galeano1985aa} The trilogy ends with the vivid image of the `tree of life':\begin{quote}
The tree of life knows that, whatever happens, the warm music spinning around it will never stop. However much death may come, however much blood may flow, the music will dance men and women as long as the air breathes them and the land plows and loves them.\autocite[p. 331]{galeano1989aa}\end{quote}In this text, community means kinship with the whole world: community entails the understanding that we are part of one another and that the natural world is part of us; that though we breathe the air, the air also breathes us. For Galeano, an individualistic, competitive society cannot truly flourish, as it disavows the synergy and reciprocity necessary for all natural growth and life. As, according to Collingwood, ``Dante express[es] what it feels like to be a Thomist'' in his literary works,\autocite[p. 295]{collingwood1938aa} Galeano expresses in his poetic history how it feels to be a socialist and environmentalist in late twentieth century Latin America. ``The poet,'' suggests Collingwood, ``converts human experience into poetry \ldots by fusing thought itself into emotion: thinking in a certain way and then \emph{expressing how it feels to think in that way}.''\autocite[p. 295][My emphasis]{collingwood1938aa}

If Galeano were writing a philosophical text, we would evaluate his work in a different way. It would be important to us whether he was right or wrong about, for example, socialism. Because this is a literary text, however, the truth we seek in it is not the truth about what a community should be, but the truth about Galeano's fine-grained cognitive-emotional experience of this concept. In thematising the concept `community', it is not necessary or obvious to write about Latin American revolutions, about water, grass, trees and chili, or to structure one's text around the narration of hundreds of tiny stories and incidents. It is Galeano specifically that finds this an apt way to proceed, at a particular moment in his emotional life which he endeavours to sincerely express through his intricate literary composition.

\section{Conclusion}
It goes without saying that more examples would need to be analysed in greater depth to properly support my overall argument. Further, while I have relied in my argument on the conception of sincerity as something allowing us to grasp the cognitive-emotional states of a particular author, a future discussion might want to revisit the points I have made in light of the complex and essential distinctions between the author, the authorial voice and the narrator(s) within a text. A more profound problem, perhaps, is the reliance of the philosophical arguments I have made upon the exposition of sweeping and potentially unwieldy theories of expression, empathy, self-knowledge, epistemic warrant and so forth. Collingwood's theory of art is idiosyncratic and contested, and needs to be defended independently if my claims are to hold. However, I would note that my argument depends chiefly on the claim that there is a certain interdependence between self-knowledge, sincerity and expression; something which might stand independently of Collingwood's broader aesthetic theory.

These limitations notwithstanding, what I have tried to indicate is that in the process of reading certain literary texts as expressive artworks, our literary-thematic appreciation constitutes an effort to reconnect an artefact with the human experience from which it arose.\footnote{I am influenced here by \autocite{dewey34}.} The arguments I have made, I hope, can lead us to entertain the possibility that a viable cognitivist theory of literature is an expressivist one.

% The signoff command also generates the bibliography, and should not be omitted.
\signoff{Katy Thomas recently completed an MA in Philosophy and Literature at the University of Warwick, following a BA in English at Sussex. Currently preparing for doctoral work, she is particularly interested in theories of expression, the political impact of the arts, and the intersection of literary aesthetics and literary theory. Her literary research is focused on South American Boom and Post-Boom literature, including the \emph{testimonio} genre. Her MA dissertation sought to question the conceptual distinction between personally expressive and politically interventionary literary works.}


\end{document}